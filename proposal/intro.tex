%!TEX root = umthsmpl.tex
\unnumberedchapter{Introduction}
People lack control over the
efforts of credit agencies and marketing firms to mine information
from financial transactions. Merchants that accept
credit and debit cards have a history of security  
failures, such as the theft of credit card data from Target~\cite{harris:2014,perlroth:2013} or the recent security breach of Equifax~\cite{equifax} exposing the personal information of millions. Centralized agencies and firms often represent a single point of attack and
 failure.

Therefore, a growing number of people have turned to decentralized {\em virtual
currencies} (VCs) such as Bitcoin~\cite{Nakamoto:2009},
Litecoin (\url{http://litecoin.org/}) and Ethereum~\cite{ETHASH} for convenience, speculation, 
or as a potential source of financial privacy and security. The benefits of virtual currencies are many: low transaction
fees, transactions over the Internet, and potentially, convenience and
privacy. While VCs offer solutions to some problems posed by centralized entities, they also introduce an additional set of concerns.

We contribute several complementary mechanisms to increase a blockchain system's security, efficiency, and transparency. Our proposed mechanisms work together or are separately deployable, and are applicable to any blockchain-based network protocol.

\section*{Contributions}
The following is a summary of the contributions in each chapter of this dissertation. 
\begin{enumerate}
\item \textbf{Analysis and improvement of algorithm for setting difficulty.} We show that the current algorithm for setting difficulty has bias and high variance, and derive an alternative estimator based on information available in the blockchain. We show that by using only the inter-arrival time of blocks (if accurate), we can estimate difficulty with no bias and lower variance. However, because many blockchain systems have weak time synchronization requirements allowing for falsification, any third-party's out-of-band record of the time cannot be secured using only information in the blockchain. Therefore, I also propose to create a second estimator based on compact {\em status reports} regularly broadcast from miners of their partial POW. These reports add no computational load to miners, and are stored neither on the blockchain nor at peers that receive them past their usefulness. They are small and can be broadcast out-of-band, for example via RSS or Twitter. Just like block headers, reports are verifiable as authentic POW by third parties. I plan to examine hybrid approaches that use both estimators, allowing for incremental deployment. 
\item \textbf{Novel method for status report and block distribution.}
\item \textbf{Evaluation of the practical feasibility of selfish mining and double-spend attacks.} Yet, all earlier studies of the economics of double-spend attacks fall short because of the simplicity of their model and resulting inability to capture the full complexity of the problem. In the present work, we derive a novel, continuous-time model for the double-spend attack.
\end{enumerate}
Our results can be used by blockchain designers to understand the consequences of setting the parameters of their difficulty algorithms. 

\section*{Collaborators}
All research activities are conducted under the supervision of Brian Levine. Preliminary work for Chapter 2 was completed in collaboration with George Bissias and Brian Levine; that for Chapter 3 was completed in collaboration with Gavin Andresen, George Bissias, Amir Houmansadr and Brian Levine; and that for Chapter 4 was completed in collaboration with Philip Thomas and Brian Levine.