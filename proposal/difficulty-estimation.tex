%!TEX root = umthsmpl.tex
\chapter{Difficulty Estimation}
\label{difficulty-estimation}

\section{POW in Blockchain Systems}
Bitcoin uses a simple POW algorithm based on cryptographic
hashing, proposed earlier by Douceur~\cite{Douceur:2002}.
Specifically, miners apply a 256-bit cryptographic hash algorithm~\cite{hashcash} to
an 80-byte {\em block header}, and the puzzle is solved if the
resulting value is less than a known {\em target}, $0<t<2^{256}$. The
header in Bitcoin consists of the Merkle root of the set of
transactions, a timestamp, the target (stored as $2^{224}/t$), a {\em
  nonce}, and the hash of the prior block's header. If the hash is not
less than the target, then a new nonce is selected to generate a new
hash (the Merkle root can be adjusted as well). This process repeats
until some miner finds a solution.
  
Each time a nonce is selected and the block
header is hashed, the miner is sampling a value from a discrete uniform
distribution with range $[0,2^{256}-1]$. The probability of solving the POW and
discovering a block is the cumulative probability of selecting a value
from $[0,t]$, which is $t/{2^{256}}$. Hence, in expectation, the
number of samples needed to discover a block is ${2^{256}/t}$. Bitcoin
adjusts the target so that on average it takes about 600 seconds to
find a block. Typically, the target is described for convenience as a
{\em difficulty}, defined to be $D=2^{224}/t$. Bitcoin's difficulty
is set once every two weeks.

\para{Ethereum.} Ethereum operates very similarly to Bitcoin. The
following differences are relevant to the context of this paper.
Ethereum miners solve a POW problem that is more complicated
than Bitcoin in an attempt to disadvantage miners with custom ASICs.
However, in the end, a miner still compares a hash value to the target.
Specifically, the number of values in the block header is larger,
resulting in a 508-byte header. It's not the hash of the header that
is compared against the target, but the hash resulting from an
Ethereum-specific algorithm called ETHASH~\cite{ETHASH}, for which the
hash of the block header is the primary input. In the end, the POW
hash value is a sample from a discrete uniform distribution with range
$[0,2^{256}-1]$, and the probability of block discovery is
${t/2^{256}}$.

A major difference of Ethereum is that the target is set such that the
expected time between blocks is 15 seconds. This setting results
in quicker confirmation times, but as a result, the probability that two
miners announce blocks within the propagation time of a block
announcement is much higher. Therefore, there are many abandoned
forks in the chain. Ethereum uses a modified version of the GHOST~\cite{Sompolinsky:2015}
protocol for selecting the main fork of the blockchain: the main chain
follows the block at each level with the most POW on its subtree.
These differences do not affect the application of our algorithms; in
fact, the presence of ommers is additional data which improves our
estimates.

%%%%%%%%%%%%%%%%%%%%%%%%%%%%%%%%%%%%%%%%
\section{Problem Statement}
Developers have resorted to ad-hoc methods for updating the difficulty in many blockchain systems. So far, there has been no previous work on analyzing the efficiency and correctness of these methods. In fact, because some blockchain systems do not accurately update difficulty, networks see enormous variance in inter-block delay. If mining power were constant in these networks, then difficulty could be kept constant. However, as seen with many blockchain systems, including a recent example with Bitcoin Cash, mining power fluctuates within these networks, requiring a readjustment of network parameters, the most important being difficulty. Therefore, in this chapter, I use the most prominent cryptocurrency, Bitcoin, as a testbed for analyzing difficulty, and then propose alternative methods that could potentially increase efficiency. 

%%%%%%%%%%%%%%%%%%%%%%%%%%%%%%%%%%%%%%%%
\section{Preliminary Work}
\subsection{Analysis of Bitcoin's Difficulty Estimation}
\para{Algorithm for setting difficulty.} Bitcoin's networks parameters are set such that a block is discovered every 10 minutes. Initially, difficulty starts at 1, and then for every 2016 blocks that are found, the timestamps of the blocks are compared to find out how much time it took to find 2016 blocks. Let $t$ denote the time in minutes it took to find 2016 blocks. Because the network is configured such that 2016 blocks must take 2 weeks (20160 minutes), the old difficulty is multiplied by $20160 / t$. If the correction factor is greater than $4$ or less than $1/4$, then $4$ or $1/4$ are used, respectively, to prevent the change from being too abrupt.
\par \noindent Bitcoin's target and difficulty are related to each other as follows: target = targetmax / difficulty = $2^{224}/{D_i}$, where $D_i$ is the $i$th time the difficulty is set~\cite{bitcoin:difficulty}. The difficulty and target are inversely proportional: when difficulty increases, the target decreases. Therefore, a smaller target makes block creation more difficult, and as the difficulty goes up, so does the expected time needed to create a block.

\para{Analysis of difficulty adjustment.}
Let $D_i$ denote the $i$th time the difficulty is set, and $X_k$ denote the number of minutes it took to generate the $k$th block after $D_i$ is set. Then we have
\begin{align}
D_{i} &=D_i \frac{10n}{ \sum_{k=1}^n X_k}.
%&= 10n^i \frac{1}{\prod_{x=0}^i t_x}
\end{align}
Let D be a sequence generated by the last equation where the first element of the sequence is $D_0 = 1$. Mining is an example of a Poisson process because, under constant
mining power, blocks are mined continuously and independently at a
constant average rate.  Therefore, $X_k \sim$ Exp$(\beta)$ with $\beta = 1/\lambda$, assuming that the miner hash rate stays constant for the 2-week period after $D_i$ is set. In this parametrization of the exponential, $\beta$ represents the survival parameter, and hence, the ratio describing the {\em expected time} it takes for one block to arrive. For example, ideally in Bitcoin $\beta=1/10$ minutes and in Ethereum $\beta=1/15$ seconds. The difficulty for the $i+1$st time given $D_i$ and $X_1, \dots, X_n$ is
\begin{align}
D_{i+1} &= 10nD_i \Bigg(\frac{1}{\sum_{k=1}^{n} X_k}\Bigg).
%&= 10n D_i \frac{1}{Y}
\end{align}

\para{The relationship between hash rate and $\boldsymbol{\beta}$.}
Given difficulty $D_i$, the expected number of hashes, $h$, needed to meet the target for a block is
\begin{align}
\EX[h] =  \frac{2^{256}-1}{T_i} = \frac{2^{256}-1}{\sfrac{2^{224}}{D_i}} = \frac{D_i(2^{256}-1)}{2^{224}}, 
\end{align}
where $T_i$ is the target set for the $i$th time. Note that $\EX[h]$ describes the \textit{total} number of expected hashes needed to discover a block, and I have observations regarding the \textit{time} it takes to generate a block. Let $r$ be the hash rate of the network in minutes (or the number of hashes per time unit), and $X = X_1, \dots, X_{n}$, where $X \sim$ Exp$(\beta)$, with $\beta = 1/\lambda$. $\lambda r$ is the expected number of hashes each time a block is created.  %We do not know the real value of $\beta$ but can use the estimator in eq. 1.
\begin{align}
\EX[h] = r \lambda &= r \frac{1}{\beta}  \\
r &= \EX[h] \beta.
\end{align}

\para{Adjusting difficulty correctly.}
For Bitcoin, where the network is expected to solve a block every 10 minutes, we can adjust the target for the $i+1$th time as follows 
\begin{align}
\frac{(2^{256}-1)}{T_{i+1}} &= 10r \\
\frac{(2^{256}-1)}{T_{i+1}} &= \frac{10(2^{256}-1)\beta}{T_i} \\
T_{i+1} &= \frac{T_i}{10\beta}.
\end{align}
Therefore, by rearranging and substituting $T_i$ with its definition using $D_i$, we can adjust the difficulty for the $i+1$th time as follows 
\begin{align}
D_{i+1} &= \frac{2^{224}}{T_{i+1}} \\
%&= \frac{2^{224}10\beta}{T_i} \\
%&= \frac{2^{224}10\beta}{\sfrac{2^{224}}{D_i}} \\
&= 10\beta D_i.
\end{align}

%\para{Expected value of difficulty.}
%We can talk about the expected value of a term in sequence D, given its preceding term and the new data we see. 
%\begin{align}
%\EX[D_{i+1} | D_{i}, X_1, \dots, X_{n}] &= \EX\bigg[10n D_i \frac{1}{Y}\bigg] \\
%&= 10n D_i \EX\bigg[\frac{1}{Y}\bigg] \\
%&= 10n D_i \Bigg(\frac{1}{\beta(n-1)}\Bigg) \\
%&= \frac{10n D_i}{\beta(n-1)}.
%\end{align}
%Refer to \url{https://stats.stackexchange.com/questions/139467/expected-value-of-y-1-x-where-x-sim-gamma} for proof. 

\para{Variance of difficulty.}
Additionally, we can also talk about the variance of a term in sequence D, given its preceding term and the new data we see. 
\begin{align}
\text{Var}(D_{i+1} | D_{i}, X_1, \dots, X_{n}) &= \text{Var}\bigg(10n D_i \frac{1}{Y}\bigg) \\
&= (10n D_i)^2 \text{Var}\bigg(\frac{1}{Y}\bigg) \\
&= (10n D_i)^2 \Bigg(\frac{1}{\beta^2(n-1)^2(n-2)}\Bigg) \\
&= \frac{(10n D_i)^2}{\beta^2(n-1)^2(n-2)}.
\end{align}
%Refer to \url{https://www.johndcook.com/inverse_gamma.pdf} and \url{https://ocw.mit.edu/courses/mathematics/18-443-statistics-for-applications-fall-2006/lecture-notes/lecture6.pdf} for explanation.

\para{Bias of difficulty.}
\begin{align}
\text{bias}(D_{i+1}|D_i,X_1,...,X_n) &= \EX[D_{i+1}|D_i,X_1,...,X_n] - D_{i+1} \\
&= \frac{10n D_i}{\beta(n-1)} - 10\beta D_i.
%&= 10D_i \Bigg(\frac{n}{\beta(n-1)} - \beta \Bigg)
\end{align}

\para{Mean squared error (MSE) of difficulty.}
\begin{align}
\text{MSE}(D_{i+1}|D_i,X_1,...,X_n) &= \text{bias}(D_{i+1}|D_i,X_1,...,X_n)^2 + \text{Var}(D_{i+1}|D_i,X_1,...,X_n) \\
&= \Bigg(\frac{10n D_i}{\beta(n-1)} - 10\beta D_i\Bigg)^2 + \frac{(10n D_i)^2}{\beta^2(n-1)^2(n-2)}.
%&= \Bigg[10D_i \Bigg(\frac{n}{\beta(n-1)} - \beta \Bigg)\Bigg]^2 + \\
%&= (10D_i)^2 \Bigg(\frac{n^2}{\beta^2(n-1)^2} - \beta^2 - \frac{2\beta n}{\beta(n-1)} \Bigg) + \frac{(10n D_i)^2}{\beta^2(n-1)^2(n-2)}
\end{align}

%%%%%%%%%%%%%%%%%%%%%%%%%%%%%%%%%%%%%%%%
\subsection{Alternative Estimation}
%In this section, we describe two blockchain-only methods of estimating
%miner hash rates. 
%%For these estimators, we treat the entire network as a single miner and a block as a status report 
%%that can only be observed at certain intervals. 
%Although these approaches have no additional network costs and do not require cooperation from miners, they are less accurate than status reports.  We then extend our techniques to allow for hash rate estimation of an individual or a subset of miners. As we show, this extension allows for the incremental deployment of status reports. 

%We would like to estimate the network hash rate, $\hat{h}$, using only
%the \emph{inter-arrival time} of mined blocks. 

\para{Estimator for $\boldsymbol{\beta}$, the expected inter-arrival time between blocks.} Let $X = X_1, \ldots, X_n$ denote the inter-arrival time between $n+1$ consecutive blocks on the blockchain. Given a consecutive sequence of $n+1$ blocks, $n$ inter-arrival times can be computed by subtracting the timestamp of each block from that of its preceding block. Note that $X \sim$ Exp$(\beta)$ with $\beta = 1/\lambda$, similar to the definition in the previous section. It it well known that the unbiased MLE estimator for $\beta$ is
\begin{align}
\hat{\beta} = \frac{\sum_{k=1}^{n} X_i}{n}.
\end{align}

\para{Adjusting difficulty.}
Using our estimation of $\beta$, we can adjust the difficulty for the $i+1$th time as follows 
\begin{align}
%\frac{D_{i+1}(2^{256}-1)}{2^{224}} &= 10r \\
%\frac{D_{i+1}(2^{256}-1)}{2^{224}} &= \frac{10D_i(2^{256}-1)\hat{\beta}}{2^{224}} \\
D_{i+1} &= 10 D_i \hat{\beta}.
\end{align}

%\para{Expected Value of New Difficulty.}
%\begin{align}
%\EX[D_{i+1} | D_{i}, X_1, \dots, X_{n}] &= \EX[10 D_i \hat{\beta}] \\
%&= 10 D_i \EX[\hat{\beta}] \\
%&= 10 D_i\beta.
%\end{align}

\para{Variance of New Difficulty.}
\begin{align}
\text{Var}(D_{i+1} | D_{i}, X_1, \dots, X_{n}) &= \text{Var}(10 D_i \hat{\beta}) \\
&= (10 D_i)^2 \text{Var}(\hat{\beta}) \\
&= \frac{(10 D_i\beta)^2}{n}.
\end{align}

\para{Bias of New Difficulty.}
\begin{align}
\text{bias}(D_{i+1}|D_i,X_1,...,X_n) &= \EX[D_{i+1}|D_i,X_1,...,X_n] - D_{i+1} \\
&= 10 D_i\beta - 10 D_i\beta \\
&= 0.
\end{align}

\para{MSE of New Difficulty.}
\begin{align}
\text{MSE}(D_{i+1}|D_i,X_1,...,X_n) &= \text{bias}(D_{i+1}|D_i,X_1,...,X_n)^2 + \text{Var}(D_{i+1}|D_i,X_1,...,X_n) \\
&= \frac{(10 D_i\beta)^2}{n}.
\end{align}

%%%%%%%%%%%%%%%%%%%%%%%%%%%%%%%%%%%%%%%%
\subsection{Comparison of Estimators}
Note that our alternative estimator has zero bias compared to Bitcoin's original estimator. Additionally, under the appropriate constraints, variance and MSE is also significantly lower.
\par \noindent \para{Variance.}
\begin{align}
\frac{(10 D_i\beta)^2}{n} \leq \frac{(10n D_i)^2}{\beta^2(n-1)^2(n-2)} \\
\frac{\beta^2}{n} \leq \frac{n^2}{\beta^2(n-1)^2(n-2)} 
\end{align}
Our estimator (LHS) has lower variance than the original estimator (RHS) for $n>2$ and $0<\beta<1$.
%Reduce[\frac{\beta^2}{n} \leq \frac{n^2}{\beta^2(n-1)^2(n-2)}, \beta<1]

\para{MSE.}
\begin{align}
\frac{(10 D_i\beta)^2}{n} \leq \Bigg(\frac{10n D_i}{\beta(n-1)} - 10\beta D_i\Bigg)^2 + \frac{(10n D_i)^2}{\beta^2(n-1)^2(n-2)}
\end{align}
Our estimator (LHS) has lower variance than the original estimator (RHS) for $n>2$, $0<\beta<1$ and $D_i \in \mathcal{R}$. \reminder{Add figure for Bitcoin, Litecoin, Ethereum, Bitcoin Cash, Ripple.}
%Reduce[\frac{(10 D \beta)^2}{n} \leq \Bigg(\frac{10n D}{\beta(n-1)} - 10 \beta D\Bigg)^2 + \frac{(10n D)^2}{\beta^2(n-1)^2(n-2)}, \beta<1]
%So, time-based estimator is indeed better.

%%%%%%%%%%%%%%%%%%%%%%%%%%%%%%%%%%%%%%%%
\section{Proposed Work}
We were able to show that our estimator performs better than Bitcoin's ad-hoc method. To extend this chapter, I propose to answer a few more key questions:
\begin{enumerate}
\item Analyze different attacks against our estimator. What happens when the timestamps on the blocks are reported inaccurately by an attacker? How much error can an attacker introduce?
\item How often should difficulty be adjusted? Should a sliding window or non-overlapping window of blocks be used?
\item Given a required error rate, what is the shortest window of time (or number of blocks) needed to estimate difficulty correctly?
\item Given that mining power is changing, how quickly can either estimator adapt to change?
\item Attackers can easily lie about the timestamps on the blocks with this new estimator. Therefore, an alternative estimator I plan to develop uses status reports, which is a block header except that the POW does not satisfy the current target. Instead, the minimum hash value in the report represents the hash found since the last block broadcast on
the chain. To be clear, each status report does not directly report the minimum hash value; instead, reports are of the input values to the POW algorithm. Because attackers can't lie about their POW, an estimator based on the minimum hash value found is safer and can be used to adjust the emergency difficulty when the network fails to produce a block for an unusually long time.
\item What are the advantages and disadvantages of status reports compared to the time-based estimator? What are the variance and MSE of the status report based estimator?
\item What are the Chernoff bounds associated with these estimators?
\end{enumerate}

%\paragraph*{Overview}

%\begin{equation}
%\arg\max_\mathbf{s} p(\mathbf{s}|u).
%\end{equation}
%
%\begin{figure}\begin{center}
%		\includegraphics[width=0.7\textwidth]{graphics/linking2}
%		\caption{The $x$-axis represents accuracy of identifying a set of traces unlinked, and the $y$-axis represents the accuracy of the same set linked based on Equation~\ref{eq:link3}. Red is a link before, and blue is a link after.
%		\label{fig:linkcdf}}
%\end{center}\end{figure}