%!TEX root = umthsmpl.tex
\chapter{Overview of Blockchain Systems}

In this chapter, we describe the basic operation of blockchains using Bitcoin Core as an example. Other cryptocurrencies such as Litecoin and Ethereum operate similarly with minor differences.

\section{Basic Operation}

\para{Accounts.} A bitcoin is a unit of currency, which is
fungible, divisible (up to eight decimal places), and recombinable.
It is measured as a balance across multiple accounts, which are
themselves manifested in \emph{addresses}.$\!$\footnote{Internally,
  Bitcoins exist only as ``unspent transaction outputs'' (UTXO), but
  users of the system think of them as balances in addresses. } Each address
comprises a stored asymmetric cryptographic key and an associated
balance of Bitcoin. The public portions of an address are the public
key and the balance of coin. When an address is involved in a
\emph{transaction} with one or more other addresses, bitcoins are
transferred among them. Addresses are explicitly pseudonyms, and not tied to a particular 
individual; further, empty addresses can be created at no cost beyond
 generating an asymmetric key pair.

\para{Adding to the blockchain.} To be added to the {\em blockchain},
transactions are broadcast by users on Bitcoin's peer-to-peer (p2p) network. A set of
\emph{miners} on the p2p network verify that each transaction is
signed correctly, does not conflict with a previous transaction, does not
move more coin than is contained in the address, and other functions.
Each miner independently agglomerates a set of valid transactions into
a candidate \emph{block} and attempts to solve a predefined
cryptographic puzzle as {\em proof-of-work} (POW), which involves data
from the candidate block and a specific {\em prior block}. The
new transactions  are only valid if they do not
conflict with the set of transactions that are contained in all
blocks that are direct ancestors.

The first miner to solve the problem broadcasts his solution to the
network, and by virtue of the solution, is able to add the block to
the ever-growing blockchain as a child of the prior block. The miners
then start over, using the newly appended blockchain and the set of
remaining transactions. The miners' incentive for {\em discovering}
a new block is a reward of coins, called the {\em coinbase},
consisting of a predetermined {\em block reward} (currently worth 12.5 Bitcoins)
and fees from transactions included in the block.

In Bitcoin Core, the POW computation is dynamically calibrated to take
approximately ten minutes per block. When transactions appear in a
block, they are \emph{confirmed}, and each subsequent block
provides additional confirmation. To announce a new block, a miner
lists all transactions contained in the new block along with a header
that contains an easily-verifiable POW solution. When a node or
miner receives a new block, he validates each transaction in the block
and the POW.

Notably, if there is a fork on the chain, honest miners always select the prior block as the last block containing the largest amount of POW. However, due to propagation delays in the network, it is possible for the
miners to receive competing (but valid) block announcements, which
bifurcates the chain, until one of the two forks is appended to
first. It is also possible and valid for a miner to receive a set of
blocks that retroactively rewrites many blocks; doing so is a
demonstration of computational work that miners accept despite the age
or depth\footnote{The \emph{depth} of a block refers to the number of blocks that follow it; the \emph{height} of a block is the number of
blocks that precede it.} of a rewritten block.

%However,
%due to propagation delays in the network, 
%miners can receive competing (but valid) block announcements, which
%bifurcates the chain, until one of the two forks is appended to first.

Any entity can elect to be a miner for Bitcoin Core, and there is no
centralized party from whom to seek approval for mining. If all miners
were to simply vote on which block should be appended to the main chain, then the mining process would fall vulnerable to a {\em Sybil
attack}~\cite{Douceur:2002}, where an attacker presents himself as multiple identities on the network. The POW puzzle addresses this
problem by performing a kind of decentralized leader election: the
miner that solves the puzzle can decide which block to append to the
chain.

\para{Full nodes.} {\em Full nodes} are peers in the network that do not mine, but do
generate, validate, and propagate transactions and blocks to other
nodes including miners. Consumers (i.e., those who purchase goods or
services) typically have no need to process and validate all
transactions, so they can instead operate \emph{simple payment
  verification} (SPV) nodes that process, store, and transmit data
involving only addresses-of-interest, which are typically addresses
they control, make payments to, or receive payments from. SPV nodes
rely on full nodes to relay transactions-of-interest.

% These SPV nodes are intended to run on resource-constrained devices,
% such as smartphones, which typically possess between 16--64 GB of
% storage shared among all applications. They are the leaves of the
% Bitcoin p2p network (i.e., they connect to only a few full nodes and
% never other SPV nodes), relying on full nodes to operate as a
% liaison between them and the miners. To remain anonymous to full
% nodes, addresses-of-interest are deposited by the SPV node into a
% Bloom filter. Bloom filters are a specially prepared array that is
% an efficient digest of the elements in a set. With the array,
% another peer can determine, with high probability, whether an
% element is contained in the set that produced the array. This filter
% is passed to neighboring full nodes that continuously monitor all
% transactions, identify transactions-of-interest using it, and
% forward those transactions to the SPV node.

\para{Bitcoin transaction consistency.} 
The main goal of the p2p network is to provide a consistent
view of blocks and unconfirmed transactions across all network peers.
Each peer maintains a local snapshot of the transactions in a memory
pool dubbed the \emph{mempool}. Blocks consist of a list of
transactions that have already (almost always) been broadcast to
miners and full nodes in the network.

%To announce a new block, a miner lists all transactions contained in
%the new block along with a header that provides an easily verifiable
%\emph{proof-of-work} (POW) solution.  When a full node or miner receives a new block, it
%validates each transaction in the block and the proof of work.

%\subsection{Bitcoin}
%\para{Accounts.} Bitcoin users store bitcoins in accounts called {\em
%  addresses}, which are created with an empty balance simply by
%generating a public/private key pair. The transfer of coin between
%addresses, via {\em transactions}, is recorded on a public ledger
%called a {\em blockchain}. Transactions are authenticated via a
% private key signature, and the balance of each account can be 
%derived from the blockchain. 
%
