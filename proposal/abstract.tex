%!TEX root = umthsmpl.tex

In this dissertation, we analyze and improve the security and performance of blockchain systems across three primary themes. In the first theme, we analyze Bitcoin Core's algorithm for setting difficulty, a network parameter that controls the inter-arrival of blocks. Fluctuations in mining power can cause uneven inter-block delays when the difficulty is not set accurately.  Mining power can change due to many reasons, including the miners' allocation of hardware and swings in the exchange rate of a currency.  For example, Bitcoin Cash saw enormous variance in mining power at its creation and the algorithm for difficulty did not easily converge. Therefore, we propose and characterize two alternatives to accurately update difficulty: one that solely uses information that is currently available, and another based on status reports that are partial blocks regularly broadcast.

Status reports add overhead into networks because they require the propagation of additional information. In a second theme, we introduce a novel method for the propagation of status reports and blocks. We show that our approach, called Graphene, improves network performance by reducing the size of blocks.

In the third theme, we analyze the practical feasibility of prominent attacks, such as double spending and selfish mining, on blockchain systems. Most analyses generally assume that the mining power of honest and malicious miners is known by an attacker. However, we show that estimation of mining power introduces error into these models. Therefore, we argue that these attacks are difficult to carry out with high precision, and use reinforcement learning techniques to realistically evaluate them when an attacker does not have full knowledge of the network's mining power.
