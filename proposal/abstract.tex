%!TEX root = umthsmpl.tex

In this thesis, I analyze and improve the security and performance of blockchain systems across three primary themes. In the first theme, I analyze blockchain algorithms for setting block discovery difficulty. Unfortunately, churn in mining power can cause uneven inter-block delays when the difficulty is not set accurately.  Mining power can change due to many reasons, including the miners� allocation of hardware and swings in the exchange rate of a currency.  For example, Bitcoin Cash has seen enormous variance in mining power since its creation and the existing algorithm for difficulty did not easily converge. I propose two alternatives to accurately update difficulty: one that solely uses information that is currently available in blockchain networks, and another based on status reports regularly broadcast from some or all miners of their partial proof-of-work (POW). Status reports can also be used for emergency difficulty adjustment, an algorithm the network resorts to when a block takes unusually long to discover.

Status reports add overhead into networks because they require the broadcast of additional information. In a second theme, I introduce a novel method of interactive set reconciliation for the distribution of status reports in order to reduce traffic.   Even without status reports, this protocol works for the efficient distribution of blocks. The approach, called Graphene, couples a Bloom filter with an IBLT. Then I evaluate performance analytically and show that Graphene blocks are always smaller and therefore network performance is improved.

In the third theme, I analyze the practical feasibility of double-spend and selfish mining attacks on blockchain systems. The hash rate of miners is the primary quantitative factor that determines the security of any POW based blockchain consensus algorithm. Most analyses generally assume that the hash rate of honest and malicious miners is known. However, I show that hash rate estimation is difficult and introduces high variance. Therefore, I argue that these double-spend and selfish mining attacks are difficult to carry out with high precision, and use reinforcement learning techniques to realistically evaluate these attacks when an attacker does not have full knowledge of the network�s mining power.
