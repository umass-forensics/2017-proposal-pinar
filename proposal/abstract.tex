%!TEX root = umthsmpl.tex

Blockchains, such as Bitcoin, Ethereum and Litecoin, are among the most successful peer-to-peer systems on the Internet. Bitcoin has been adopted more widely for e-commerce than any previous digital currency. Ethereum is also quickly gaining prominence as a blockchain that runs Turing-complete distributed applications. Even though these systems have advantages, including decentralized operation, there also exist many limitations that decrease the security and privacy of blockchains.

In this thesis, I analyze and improve the security of blockchain systems. In the first chapter, I analyze a blockchain system's algorithm for setting block discovery difficulty. Difficulty is updated glacially in most systems (e.g., every two weeks in Bitcoin). However, the of churn of mining power can cause problems when the difficulty is not set often. Mining power can change due to miners' updating to new hardware, diurnal changes in electricity rates, or swings in the exchange rate of a currency. For example, Bitcoin Cash has seen enormous variance in mining power since its creation. I propose two alternatives to accurately update difficulty: one that solely uses information that is currently available in blockchain networks, and another based on status reports regularly broadcast from some or all miners of their partial proof-of-work (POW). Status reports can also be used for emergency difficulty adjustment, an algorithm the network resorts to when a block takes unusually long to discover.

Status reports add overhead into networks because they require the broadcast of additional information. In order to reduce traffic, in the second chapter, I introduce a novel method of interactive set reconciliation for the distribution of status reports. Even without status reports, this protocol works for the efficient distribution of blocks. The approach, called Graphene, couples a Bloom filter with an IBLT. Then I evaluate performance analytically and show that Graphene blocks are always smaller.

In the third chapter, I analyze the feasibility of double-spend and selfish mining attacks on blockchain systems. The hash rate of miners is the primary quantitative factor that determines the security of any POW based blockchain consensus algorithm. Most analyses generally assume that the hash rate of honest and malicious miners is known. However, I show that hash rate estimation is difficult and introduces high variance. Therefore, I argue that these double-spend and selfish mining attacks are difficult to carry out with high precision, and use reinforcement learning techniques to realistically evaluate these attacks without full knowledge of mining power.
