%!TEX root = umthsmpl.tex

As people carry devices that are more persistently connected to the Internet, their location privacy becomes an increasing concern. The crux of this problem is the link between a mobile device user's identity, and the locations that the identity has visited. 
%If a mobile device is not linked to a particular identity, then there is no location privacy risk. 
A user may turn of her device completely to drastically reduce her location privacy risk; however, her device is no longer useful, and her location may still be inferred if she is traveling with somebody who has not taken the same precautions. 
I demonstrate the following: (1) a unified model of location profiling, privacy, and utility; (2) two studies where a user's location may be revealed without her permission; and (3) a practical mobile connectivity framework for preserving location privacy is still susceptible to location profiling.

First, I present a mathematical model of geolocation privacy risk. This model uses trajectory linking and location profiling to de-anonymize sequences of locations back to a user. I plan to validate this model by collecting real-world data and synthesizing a dataset.

Subsequently, I investigate two attackers: web services and advertisers. First, I present a study where location is inferred by a streaming web service, who can monitor the bandwidth changes a phone incurs while traveling along a path. The attacker in this case is anyone who is able to send a constant stream of data to the phone. This study demonstrated that with 70\% accuracy, a classifier could determine on which geographic route among 8 a cell phone is traveling. Second, I explore the possibilities open to an attacker who places geolocation code in a mobile advertisement to uniquely identify a user associated with an advertising identifier. I further demonstrate that users with the advertising identifier disabled may be at risk if they travel regularly with a group of people.

Finally, I look at mechanisms to thwart a cell service provider determining a user's location. Providers can trivially geolocate users in a typical cell network using cell tower triangulation. Using our framework, a privacy-conscious provider could provide service and accept payment without knowing a user's identity. I also present a method that would allow users to share or swap their mobile identifiers. In these cases, location profiling is mitigated but still possible. I present a model of utility and risk where a user can determine based on their own knowledge of location profiling how likely they would be identified among other users if they made or accepted a phone call. I plan to develop an app to locally collect data about a user and halt certain functions on the phone when privacy is at risk.

With the modern ubiquity of mobile devices, the ability of a user to be located at most times of the day in their lives is a new force to contend with. The convenience of staying connected constantly comes with a risk of being located or identified. In this thesis, I plan to make contributions that quantify risk, and make this information to be transparent to users, so that they can make informed decisions about whether to remain connected.